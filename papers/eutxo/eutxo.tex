\documentclass[runningheads]{llncs}

% 4th Workshop on Trusted Smart Contracts: https://fc20.ifca.ai/wtsc/cfp.html
%
%   Abstract Registration (preferably)	December 9, 2019
%   Paper Submission Deadline	December 12, 2019
%   Early Author Notification	January 4, 2020
%   Late Abstract Registration	January 5, 2020
%   Late Submission Deadline	January 7, 2020
%   Late Author Notification	January 20, 2020
%
% LNCS: 15 pages including references and appendices

\usepackage{url}

\usepackage{graphicx}

% If you use the hyperref package, please uncomment the following line
% to display URLs in blue roman font according to Springer's eBook style:
%% kwxm: I did as it said, but then had to add \usepackage{url}, which
%% wasn't there before (but \url worked anyway).
\renewcommand\UrlFont{\color{blue}\rmfamily}

\usepackage[dvipsnames]{xcolor}
\usepackage{verbatim}
\usepackage{alltt}
\usepackage{etoolbox}
\usepackage{paralist}

\usepackage{float}

\usepackage[cmex10]{amsmath}
\usepackage{amssymb}
\usepackage{stmaryrd}
%\usepackage{amsthm}
\usepackage{proof}

\newcommand{\red}[1]{\textcolor{red}{#1}}

\usepackage{todonotes}
%\usepackage[disable]{todonotes}
\newcommand{\todochak}[1]{\todo[inline,color=purple!40,author=chak]{#1}}
\newcommand{\todompj}[1]{\todo[inline,color=yellow!40,author=Michael]{#1}}
\newcommand{\todokwxm}[1]{\todo[inline,color=blue!20,author=kwxm]{#1}}
\newcommand{\todojm}[1]{\todo[inline,color=purple!40,author=Jann]{#1}}
\newcommand{\todor}[1]{\todo[inline,color=orange!40,author=Orestis]{#1}}
\newcommand{\todojc}[1]{\todo[inline,color=gray!40,author=James]{#1}}

%% ... plus other authors.

\usepackage[colorlinks=true,linkcolor=MidnightBlue,citecolor=ForestGreen,urlcolor=Plum]{hyperref}
%% ^ To make the links a bit less garish.  Delete this to return to normal.

\newcommand\site[1]{\footnote{\url{#1}}}
%% ^ footnote URLs

%% A figure with rules above and below.
\newcommand\rfskip{7pt}
\newenvironment{ruledfigure}[1]{\begin{figure}[#1]\hrule\vspace{\rfskip}}{\vspace{\rfskip}\hrule\end{figure}}

\renewcommand{\i}{\textit}  % Just to speed up typing: replace these in the final version
\renewcommand{\t}{\texttt}  % Just to speed up typing: replace these in the final version
\newcommand{\s}{\textsf}  % Just to speed up typing: replace these in the final version
\newcommand{\msf}[1]{\ensuremath{\mathsf{#1}}}
\newcommand{\mi}[1]{\ensuremath{\mathit{#1}}}


%% Various text macros
\newcommand{\true}{\textsf{true}}
\newcommand{\false}{\textsf{false}}

\newcommand{\hash}[1]{\ensuremath{#1^{\#}}}

\newcommand{\List}[1]{\ensuremath{\s{List}[#1]}}
\newcommand{\Set}[1]{\ensuremath{\s{Set}[#1]}}
\newcommand{\Map}[2]{\ensuremath{\s{Map}[#1,#2]}}
\newcommand{\Interval}[1]{\ensuremath{\s{Interval}[#1]}}
\newcommand{\extended}[1]{#1^\updownarrow}

\newcommand{\script}{\ensuremath{\s{Script}}}
\newcommand{\scriptAddr}{\msf{scriptAddr}}
\newcommand{\ctx}{\ensuremath{\s{Context}}}
\newcommand{\toCtx}{\msf{toContext}}

\newcommand{\toData}{\msf{toData}}
\newcommand{\fromData}{\msf{fromData}}

% Macros for eutxo things.
\newcommand{\TxId}{\ensuremath{\s{TxId}}}
\newcommand{\txId}{\msf{txId}}
\newcommand{\txrefid}{\mi{id}}
\newcommand{\Address}{\ensuremath{\s{Address}}}
\newcommand{\DataHash}{\ensuremath{\s{DataHash}}}
\newcommand{\idx}{\mi{index}}
\newcommand{\inputs}{\mi{inputs}}
\newcommand{\outputs}{\mi{outputs}}
\newcommand{\forge}{\mi{forge}}
\newcommand{\fee}{\mi{fee}}
\newcommand{\addr}{\mi{addr}}
\newcommand{\val}{\mi{value}}  %% \value is already defined

\newcommand{\validator}{\mi{validator}}
\newcommand{\redeemer}{\mi{redeemer}}
\newcommand{\datum}{\mi{datum}}
\newcommand{\datumHsh}{\mi{datumHash}}
\newcommand{\datumWits}{\mi{datumWitnesses}}
\newcommand{\hashData}{\msf{dataHash}}
\newcommand{\validityInterval}{\mi{validityInterval}}
\newcommand{\Data}{\ensuremath{\s{Data}}}

\newcommand{\outputref}{\mi{outputRef}}
\newcommand{\txin}{\mi{in}}
\newcommand{\id}{\mi{id}}
\newcommand{\lookupTx}{\msf{lookupTx}}
\newcommand{\currentTick}{\msf{currentTick}}
\newcommand{\getSpent}{\msf{getSpentOutput}}

\newcommand{\tick}{\ensuremath{\s{Tick}}}
\newcommand{\spent}{\msf{spentOutputs}}
\newcommand{\unspent}{\msf{unspentOutputs}}
\newcommand{\txunspent}{\msf{unspentTxOutputs}}
\newcommand{\eutxotx}{\msf{Tx}}

\newcommand{\qty}{\ensuremath{\s{Quantity}}}
\newcommand{\token}{\ensuremath{\s{Token}}}
\newcommand{\currency}{\ensuremath{\s{CurrencyId}}}
\newcommand{\nativeCur}{\ensuremath{\mathrm{nativeC}}}
\newcommand{\nativeTok}{\ensuremath{\mathrm{nativeT}}}
\newcommand{\injectNative}{\ensuremath{\mathrm{inject}}}

\newcommand{\qtymap}{\ensuremath{\s{Quantities}}}

\newcommand\B{\ensuremath{\mathbb{B}}}
\newcommand\N{\ensuremath{\mathbb{N}}}
\newcommand\Z{\ensuremath{\mathbb{Z}}}
\renewcommand\H{\ensuremath{\mathbb{H}}}
%% \H is usually the Hungarian double acute accent
\newcommand{\emptyBs}{\ensuremath{\emptyset}}

\newcommand{\emptymap}{\ensuremath{\{\}}}

% multisig
\newcommand{\msc}{\mathrm{msc}}
\newcommand{\sig}{\mathit{sig}}
\newcommand{\sigs}{\mathit{sigs}}
\newcommand{\auth}{\mathrm{auth}}
\newcommand{\Holding}{\msf{Holding}}
\newcommand{\Collecting}[2]{\msf{Collecting}(#1, #2)}
\newcommand{\Propose}[1]{\msf{Propose}(#1)}
\newcommand{\Add}[1]{\msf{Add}(#1)}
\newcommand{\Cancel}{\msf{Cancel}}
\newcommand{\Pay}{\msf{Pay}}

% Agda code commit hash
\newcommand\AgdaCommit{a1574e6}

% For anonymisation
\newtoggle{anonymous}
\togglefalse{anonymous}
\iftoggle{anonymous}{
\newcommand{\Bcc}{CHAIN}
\newcommand{\Zerepoch}{LANG}
\newcommand{\GitUser}{anonymous-agda}
}{
\newcommand{\Bcc}{Bcc}
\newcommand{\Zerepoch}{Zerepoch Core}
\newcommand{\GitUser}{omelkonian}
}
\newcommand{\anonymize}[1]{\iftoggle{anonymous}{}{#1}}

% Names, for consistency
\newcommand{\UTXO}{UTXO}
\newcommand{\EUTXO}{E\UTXO{}}
\newcommand{\ExUTXO}{Extended \UTXO{}}
\newcommand{\CEM}{CEM}

% ------

\newcommand\isFinal{\msf{isFinal}}
\newcommand\step{\msf{step}}
\newcommand\satisfies{\msf{satisfies}}
\newcommand\checkOutputs{\msf{checkOutputs}}

\newcommand\mkValidator[1]{\msf{validator}_#1}
\newcommand\Sim[2]{\ensuremath{
#1 \sim #2
}}
\newcommand\CStep[3]{\ensuremath{
#1 \xrightarrow{\hspace{5pt} #2 \hspace{5pt}} (#1' , #3)
}}
\newcommand\LStep[2]{\ensuremath{
#1 \xrightarrow{\hspace{5pt} #2 \hspace{5pt}} #1'
}}
\newcommand\txeq{tx^\equiv}

%% ------------- Start of document ------------- %%

\sloppy
\begin{document}

\title{The \ExUTXO{} Model}

% \author{Anonymous\inst{1} \and
% Anonymous\inst{2,3}}

% First names are abbreviated in the running head.
% If there are more than two authors, 'et al.' is used.

\iftoggle{anonymous}{
\author{Anonymous}
\authorrunning{Anonymous et al.}
}{
% author order: alphabetical
\author{
  Manuel M.T. Chakravarty\inst{1}
  \and
  James Chapman\inst{1}
  \and
  Kenneth MacKenzie\inst{1}
  \and
  Orestis Melkonian\inst{1,2}
  \and
  Michael Peyton Jones\inst{1}
  \and
  Philip Wadler\inst{2}
}

\authorrunning{Chakravarty et al.}

\institute{
  The Blockchain Co.,
  \email{\{manuel.chakravarty, james.chapman, kenneth.mackenzie, orestis.melkonian, michael.peyton-jones\}@blockchain-company.io}
  \and
  University of Edinburgh, \email{orestis.melkonian@ed.ac.uk}, \email{wadler@inf.ed.ac.uk}
}
}

\maketitle

\begin{abstract}
  Bitcoin and Ethereum, hosting the two currently most valuable and
  popular cryptocurrencies, use two rather different ledger models,
  known as the \emph{\UTXO{} model} and the \emph{account model},
  respectively. At the same time, these two public blockchains differ
  strongly in the expressiveness of the smart contracts that they
  support. This is no coincidence. Ethereum chose the account model
  explicitly to facilitate more expressive smart contracts. On the
  other hand, Bitcoin chose \UTXO{} also for good reasons, including
  that its semantic model stays simple in a complex concurrent and
  distributed computing environment. This raises the question of
  whether it is possible to have expressive smart contracts, while
  keeping the semantic simplicity of the \UTXO{} model.

  In this paper, we answer this question affirmatively. We present
  \emph{\ExUTXO{}} (\emph{\EUTXO{}}), an extension to Bitcoin's \UTXO{} model
  that supports a substantially more
  expressive form of \emph{validation scripts}, including scripts that
  implement general state machines and enforce invariants across
  entire transaction chains.

  To demonstrate the power of this model, we also introduce a form of state
  machines suitable for execution on a ledger, based on Mealy machines
  and called Constraint Emitting Machines (\CEM{}).  We formalise \CEM{}s,
  show how to compile them to \EUTXO{}, and show a \emph{weak bisimulation}
  between the two systems. All of our work is formalised using the Agda
  proof assistant.
\end{abstract}

\keywords{blockchain \and \UTXO{} \and functional programming \and state machines.}

\section{Introduction}

Bitcoin, the most widely known and most valuable cryptocurrency, uses
a graph-based ledger model built on the concept of \emph{\UTXO{}s} (\emph{unspent
  transaction outputs})~\cite{formal-model-of-bitcoin-transactions,Zahnentferner18-UTxO}. Individual \emph{transactions} consist of a list of \emph{inputs} and a list of \emph{outputs}, where outputs represent a specific \emph{value} (of a cryptocurrency) that is available to be spent by inputs of subsequent transactions. Each output can be spent by (i.e., connect to) exactly one input. Moreover, we don't admit cycles in these connections, and hence we can regard a collection of transactions spending from each other as a directed acyclic graph, where a transaction with $m$ inputs and $n$ outputs is represented by a node in the graph with $m$ edges in and $n$ edges out.
The sum of the values consumed by a transaction's inputs must equal the sum of the values provided by its outputs, thus value is conserved.

Whether an output can be consumed by an input is determined by a function $\nu$ attached to the output, which we call the output's \emph{validator}. A transaction input proves its eligibility to spent an output by providing a \emph{redeemer} object $\rho$, such that \(\nu(\rho) = \true\); redeemers are often called \emph{witnesses} in Bitcoin. In the simplest case, the redeemer is a cryptographic hash of the spending transaction signed by an authorised spender's private key, which is checked by the validator, which embeds the corresponding public key. More sophisticated protocols are possible by using more complex validator functions and redeemers --- see \cite{bitml} for a high-level model of what is possible with the functionality provided by Bitcoin.

The benefit of this graph-based approach to a cryptocurrency ledger is that it plays well with the concurrent and distributed nature of blockchains. In particular, it forgoes any notion of shared mutable state, which is known to lead to highly complex semantics in the face of concurrent and distributed computations involving that shared mutable state.

Nevertheless, the \UTXO{} model, generally, and Bitcoin, specifically, has been criticised for the limited expressiveness of programmability achieved by the validator concept. In particular, Ethereum's \emph{account-based ledger} and the associated notion of \emph{contract accounts} has been motivated by the desire to overcome those limitations. Unfortunately, it does so by introducing a notion of shared mutable state, which significantly complicates the semantics of contract code. In particular, contract authors need to understand the subtleties of this semantics or risk introducing security issues (such as the vulnerability to recursive contract invocations that led to the infamous DAO attack~\cite{DAO-attack}).

\paragraph{Contributions.}

The contribution of the present paper is to propose an extension to the basic \UTXO{} ledger model, which
\begin{inparaenum}[(a)]
\item provably increases expressiveness, while simultaneously
\item preserving the dataflow properties of the \UTXO{} graph; in particular, it forgoes introducing any notion of shared mutable state.
\end{inparaenum}
More specifically, we make the following contributions:
%
\begin{itemize}
\item We propose the \emph{\EUTXO{} model}, informally in Section~\ref{sec:informal-eutxo} and formally in Section~\ref{sec:formal-model}.
\item We demonstrate that the \EUTXO{} model supports the implementation of a specific form of state machines (\emph{Constraint Emitting Machines}, or \CEM{}s), which the basic \UTXO{} model does not support, in Section~\ref{sec:expressiveness}.
\item We provide formalisations of both the \EUTXO{} model
  and Constraint Emitting Machines. We prove a weak bisimulation
  between the two using the Agda proof
  assistant\site{https://github.com/\GitUser/formal-utxo/tree/\AgdaCommit}, building on
  previous work by Melkonian et al.~\cite{formal-eutxo}.
\end{itemize}

\noindent Section~\ref{sec:related} summarises related work.

The \EUTXO{} model will be used in the ledger of \Bcc{}, a major blockchain
system currently being developed by The Blockchain Co..
It also provides the foundation of Bcc's smart contract platform
\emph{Zerepoch}\site{https://github.com/The-Blockchain-Company/zerepoch}, which includes a small
functional programming language \emph{\Zerepoch{}} which is used to implement \script{}s.
Although a technical description of Bcc itself is beyond the scope of this paper,
one can try out the Zerepoch Platform in an online playground.\site{https://prod.playground.zerepoch.tbcodev.io/}

Other future work includes a formal comparison of \EUTXO{} with Ethereum's account-based model.

\iftoggle{anonymous}{
\paragraph{Anonymisation.}

For the purposes of submission, we have anonymised the following names:
\begin{itemize}
\item \Bcc{}: a major blockchain system.
\item \Zerepoch{}: a small functional programming language.
\end{itemize}
}{}

\input{informal-eutxo.tex}
\input{formal-model.tex}
\input{formal-verification.tex}
\input{related-work.tex}

\bibliographystyle{splncs04}
\bibliography{eutxo}

\end{document}
