%
% ISoLA 2020
%
% http://isola-conference.org/isola2020/
%
% 31 May 2020     Paper Submission
% 29 June 2020    Reports and Notification
% 02 August 2020  Final Versions
%
% Conference: Rhodes, October 26th - October 30th 2020
%
% "The expected length is around 15 pages with some flexibility in case of need."
%

% There doesn't appear to be any anonymity requirement

\documentclass[runningheads]{llncs}
\usepackage{lineno}
%\linenumbers

% correct bad hyphenation here
\hyphenation{}

%\usepackage{natbib}
\usepackage{url}

% *** MATHS PACKAGES ***
%
\usepackage[cmex10]{amsmath}
\usepackage{amssymb}
\usepackage{stmaryrd}
%% \usepackage{amsthm}
\usepackage{proof}


% *** ALIGNMENT PACKAGES ***
%
\usepackage{array}
\usepackage{float}  %% Try to improve placement of figures.  Doesn't work well with subcaption package.
\usepackage{subcaption}
\usepackage{caption}

\usepackage{geometry}
\usepackage{listings}
\usepackage[dvipsnames]{xcolor}
\usepackage{verbatim}
\usepackage{alltt}
\usepackage{paralist}

\usepackage{todonotes}
%\usepackage[disable]{todonotes}

% This has to go at the end of the packages.
\usepackage[colorlinks=true,linkcolor=MidnightBlue,citecolor=ForestGreen,urlcolor=Plum]{hyperref}

\newcommand\ifootnote[1]{(#1)}
%% ^ inline footnote

\newcommand\site[1]{\footnote{\url{#1}}}
%% ^ footnote URL

% Tikz
\usepackage{tikz}
\usetikzlibrary{arrows,positioning,matrix,fit,decorations.text}

% Stuff for splitting figures over page breaks
%\DeclareCaptionLabelFormat{continued}{#1~#2 (Continued)}
%\captionsetup[ContinuedFloat]{labelformat=continued}

% *** MACROS ***

\newcommand{\todochak}[1]{\todo[inline,color=purple!40,author=chak]{#1}}
\newcommand{\todompj}[1]{\todo[inline,color=yellow!40,author=Michael]{#1}}
\newcommand{\todokwxm}[1]{\todo[inline,color=blue!20,author=kwxm]{#1}}
\newcommand{\todojm}[1]{\todo[inline,color=purple!40,author=Jann]{#1}}
\newcommand{\todor}[1]{\todo[inline,color=red!40,author=Orestis]{#1}}
\newcommand{\todojmc}[1]{\todo[inline,color=green!40,author=James]{#1}}

\newcommand{\red}[1]{\textcolor{red}{#1}}
\newcommand{\redfootnote}[1]{\red{\footnote{\red{#1}}}}
\newcommand{\blue}[1]{\textcolor{blue}{#1}}
\newcommand{\bluefootnote}[1]{\blue{\footnote{\blue{#1}}}}

%% A version of ^{\prime} for use in text mode
\makeatletter
\DeclareTextCommand{\textprime}{\encodingdefault}{%
  \mbox{$\m@th'\kern-\scriptspace$}%
}
\makeatother

\renewcommand{\i}{\textit}  % Just to speed up typing: replace these in the final version
\renewcommand{\t}{\texttt}  % Just to speed up typing: replace these in the final version
\newcommand{\s}{\textsf}  % Just to speed up typing: replace these in the final version
\newcommand{\msf}[1]{\ensuremath{\mathsf{#1}}}
\newcommand{\mi}[1]{\ensuremath{\mathit{#1}}}

%% A figure with rules above and below.
\newcommand\rfskip{7pt}
%\newenvironment{ruledfigure}[1]{\begin{figure}[#1]\hrule\vspace{\rfskip}}{\vspace{\rfskip}\hrule\end{figure}}
\newenvironment{ruledfigure}[1]{\begin{figure}[#1]}{\end{figure}}

%% Various text macros
\newcommand{\true}{\ensuremath{\mathsf{true}}} % \textsf becomes slanted in math mode
\newcommand{\false}{\ensuremath{\textsf{false}}}

\newcommand{\hash}[1]{\ensuremath{#1^{\#}}}

\newcommand{\List}[1]{\ensuremath{\s{List}[#1]}}
\newcommand{\Set}[1]{\ensuremath{\s{Set}[#1]}}
\newcommand{\FinSet}[1]{\ensuremath{\s{FinSet}[#1]}}
\newcommand{\Interval}[1]{\ensuremath{\s{Interval}[#1]}}
\newcommand{\FinSup}[2]{\ensuremath{\s{FinSup}[#1,\linebreak[0]#2]}}
% ^ \linebreak is to avoid a bad line break when we talk about finite
% maps.  We may be able to remove it in the final version.

\newcommand{\supp}{\msf{supp}}

\newcommand{\Script}{\ensuremath{\s{Script}}}
\newcommand{\scriptAddr}{\msf{scriptAddr}}
\newcommand{\ctx}{\ensuremath{\s{Context}}}
\newcommand{\vlctx}{\ensuremath{\s{ValidatorContext}}}
\newcommand{\mpsctx}{\ensuremath{\s{PolicyContext}}}
\newcommand{\toData}{\ensuremath{\s{toData}}}
\newcommand{\toTxData}{\ensuremath{\s{toTxData}}}
\newcommand{\fromData}{\msf{fromData}}

\newcommand{\mkContext}{\ensuremath{\s{mkContext}}}
\newcommand{\mkVlContext}{\ensuremath{\s{mkValidatorContext}}}
\newcommand{\mkMpsContext}{\ensuremath{\s{mkPolicyContext}}}

\newcommand{\applyScript}[1]{\ensuremath{\llbracket#1\rrbracket}}
\newcommand{\applyMPScript}[1]{\ensuremath{\llbracket#1\rrbracket}}

% Macros for eutxo things.
\newcommand{\tx}{\mi{tx}}
\newcommand{\TxId}{\ensuremath{\s{TxId}}}
\newcommand{\txId}{\msf{txId}}
\newcommand{\txrefid}{\mi{id}}
\newcommand{\Address}{\ensuremath{\s{Address}}}
\newcommand{\DataHash}{\ensuremath{\s{DataHash}}}
\newcommand{\hashData}{\msf{dataHash}}
\newcommand{\idx}{\mi{index}}
\newcommand{\inputs}{\mi{inputs}}
\newcommand{\outputs}{\mi{outputs}}
\newcommand{\forge}{\mi{forge}}
\newcommand{\forgeScripts}{\mi{forgeScripts}}
\newcommand{\sigs}{\mi{sigs}}
\newcommand{\fee}{\mi{fee}}
\newcommand{\addr}{\mi{addr}}
\newcommand{\val}{\mi{value}}  %% \value is already defined

\newcommand{\validator}{\mi{validator}}
\newcommand{\redeemer}{\mi{redeemer}}
\newcommand{\datum}{\mi{datum}}
\newcommand{\datumHash}{\mi{datumHash}}
\newcommand{\datumWits}{\mi{datumWitnesses}}
\newcommand{\Data}{\ensuremath{\s{Data}}}
\newcommand{\Input}{\ensuremath{\s{Input}}}
\newcommand{\Output}{\ensuremath{\s{Output}}}
\newcommand{\OutputRef}{\ensuremath{\s{OutputRef}}}
\newcommand{\Signature}{\ensuremath{\s{Signature}}}
\newcommand{\Ledger}{\ensuremath{\s{Ledger}}}

\newcommand{\outputref}{\mi{outputRef}}
\newcommand{\outputrefs}{\mi{outputRefs}}
\newcommand{\txin}{\mi{in}}
\newcommand{\id}{\mi{id}}
\newcommand{\lookupTx}{\msf{lookupTx}}
\newcommand{\getSpent}{\msf{getSpentOutput}}

\newcommand{\Tick}{\ensuremath{\s{Tick}}}
\newcommand{\currentTick}{\msf{currentTick}}
\newcommand{\spent}{\msf{spentOutputs}}
\newcommand{\unspent}{\msf{unspentOutputs}}
\newcommand{\txunspent}{\msf{unspentTxOutputs}}
\newcommand{\eutxotx}{\msf{Tx}}

\newcommand{\Quantity}{\ensuremath{\s{Quantity}}}
\newcommand{\Asset}{\ensuremath{\s{Asset}}}
\newcommand{\Policy}{\ensuremath{\s{Policy}}}
\newcommand{\Quantities}{\ensuremath{\s{Quantities}}}
\newcommand{\nativeCur}{\ensuremath{\mathrm{nativeC}}}
\newcommand{\nativeTok}{\ensuremath{\mathrm{nativeT}}}

\newcommand\B{\ensuremath{\mathbb{B}}}
\newcommand\N{\ensuremath{\mathbb{N}}}
\newcommand\Z{\ensuremath{\mathbb{Z}}}
\renewcommand\H{\ensuremath{\mathbb{H}}}
%% \H is usually the Hungarian double acute accent
\newcommand{\emptyBs}{\ensuremath{\emptyset}}

\newcommand{\emptymap}{\ensuremath{\{\}}}

\newcommand{\outRef}{\mi{out}_{\mi{ref}}}
\newcommand{\fps}{\mi{fps}}
\newcommand{\fpss}{\mi{fpss}}

\usepackage{etoolbox}

\newcommand{\Bcc}{Bcc}
\newcommand{\ZerepochCore}{Zerepoch Core}
\newcommand{\GitUser}{omelkonian}

% Macros for formalisation
\newcommand\agdaRepo{https://github.com/omelkonian/formal-utxo/tree/2d32}
\newcommand\ra{\rightarrow}
\newcommand\nft{\blacklozenge}
\newcommand\NF{\s{NonFungible}}
\newcommand\Trace{\s{Trace}}
\newcommand\Provenance{\s{Provenance}}
\newcommand\provenance{\s{provenance}}
\newcommand\isFinal{\msf{isFinal}}
\newcommand\step{\msf{step}}
\newcommand\satisfies{\msf{satisfies}}
\newcommand\checkOutputs{\msf{checkOutputs}}
\newcommand\initial{\msf{initial}}
\newcommand\extract{\msf{extract}}

\newcommand\All{\msf{All}}
\newcommand\AllS{\msf{All}^\s{S}}

\newcommand\mkValidator[1]{\msf{validator}_#1}
\newcommand\mkPolicy[1]{\msf{policy}_#1}
\newcommand\valC{\mkValidator{\mathcal{C}}}
\newcommand\polC{\mkPolicy{\mathcal{C}}}

\newcommand\CStepArrow[3]{\ensuremath{
#1 \xrightarrow{\hspace{5pt} #2 \hspace{5pt}} (#1' , #3)
}}

\newcommand\txeq{tx^\equiv}
\newcommand\Sim[2]{\ensuremath{
#1 \sim #2
}}
\newcommand\CStep[1]{\ensuremath{
 #1 \xrightarrow{\hspace{5pt} i \hspace{5pt}} (#1' , \txeq)
%%\textsf{step}\, #1\, i \equiv \textsf{just}\, #1'
}}
\newcommand\LStep[2]{\ensuremath{
#1 \xrightarrow{\hspace{5pt} #2 \hspace{5pt}} #1'
}}
\newcommand\transTo{\ensuremath{
    \rightsquigarrow^*
}}
\newcommand\transToN{\ensuremath{
    \rightsquigarrow^{+}
}}

% multisig
\newcommand{\msc}{\mathrm{msc}}
\newcommand{\sig}{\mathit{sig}}
%\newcommand{\sigs}{\mathit{sigs}}
\newcommand{\auth}{\mathrm{auth}}
\newcommand{\Holding}{\msf{Holding}}
\newcommand{\Collecting}[4]{\msf{Collecting}(#1, #2, #3, #4)}
\newcommand{\Propose}[3]{\msf{Propose}(#1, #2, #3)}
\newcommand{\Add}[1]{\msf{Add}(#1)}
\newcommand{\Cancel}{\msf{Cancel}}
\newcommand{\Pay}{\msf{Pay}}

% Names, for consistency
\newcommand{\UTXO}{UTXO}
\newcommand{\EUTXO}{E\UTXO{}}
\newcommand{\ExUTXO}{Extended \UTXO{}}
\newcommand{\CEM}{CEM}
\newcommand{\UTXOma}{\UTXO$_{\textrm{ma}}$}
\newcommand{\EUTXOma}{\EUTXO$_{\textrm{ma}}$}

\newcommand{\diffBox}[1]{\text{\colorbox{lightgray}{\(#1\)}}}

% relaxed float placement
\renewcommand{\topfraction}{.95}
\renewcommand{\bottomfraction}{.7}
\renewcommand{\textfraction}{.15}
\renewcommand{\floatpagefraction}{.66}
\renewcommand{\dbltopfraction}{.66}
\renewcommand{\dblfloatpagefraction}{.66}
\setcounter{topnumber}{9}
\setcounter{bottomnumber}{9}
\setcounter{totalnumber}{20}
\setcounter{dbltopnumber}{9}

%% ------------- Start of document ------------- %%

\begin{document}

\title{Native Custom Tokens in the Extended \UTXO{} Model}

% First names are abbreviated in the running head.
% If there are more than two authors, 'et al.' is used.

\author{
  Manuel M.T. Chakravarty\inst{1}
  \and
  James Chapman\inst{1}
  \and
  Kenneth MacKenzie\inst{1}
  \and
  Orestis Melkonian\inst{1,2}
  \and
  Jann M\"uller\inst{1}
  \and
  Michael Peyton Jones\inst{1}
  \and
  Polina Vinogradova\inst{1}
  \and
  Philip Wadler\inst{1,2}
}

\authorrunning{Chakravarty et al.}

\institute{
  The Blockchain Co.,
  \email{firstname.lastname@blockchain-company.io}
  \and
  University of Edinburgh,
  \email{orestis.melkonian@ed.ac.uk, wadler@inf.ed.ac.uk}
}

\maketitle

\begin{abstract}
  \emph{User-defined tokens} ---both fungible ERC-20 and non-fungible ERC-721 tokens--- are central to the majority of contracts deployed on Ethereum. User-defined tokens are \emph{non-native} on Ethereum; i.e., they are not directly supported by the ledger, but require custom code. This makes them unnecessarily inefficient, expensive, and complex.

  The \emph{Extended UTXO Model} (\kern-0.5pt\emph{\EUTXO})~\cite{eutxo-1-paper} has been introduced as a generalisation of Bitcoin-style \UTXO\ ledgers, allowing support of more expressive smart contracts, approaching the functionality available to contracts on Ethereum. Specifically, a bisimulation argument established a formal relationship between the \EUTXO\ ledger and a general form of state machines. Nevertheless, \emph{transaction outputs} in the \EUTXO\ model lock integral quantities of a single native cryptocurrency only, just like Bitcoin.

  In this paper, we study a generalisation of the \EUTXO\ ledger model with \emph{native} user-defined tokens. Following the approach proposed in a companion paper~\cite{plain-multicurrency} for the simpler case of plain Bitcoin-style \UTXO\ ledgers, we generalise transaction outputs to lock not merely coins of a single cryptocurrency, but entire \emph{token bundles,} including custom tokens whose forging is controlled by \emph{forging policy scripts.} We show that this leads to a rich ledger model that supports a broad range of interesting use cases.

  Our main technical contribution is a formalisation of the multi-asset \EUTXO\ ledger in Agda, which we use to establish that the ledger with custom tokens is strictly more expressive than the original \EUTXO\ ledger. In particular, we state and prove a transfer result for inductive and temporal properties from state machines to the multi-asset \EUTXO\ ledger, which was out of scope for the single-currency \EUTXO\ ledger. In practical terms, the resulting system is the basis for the smart contract system of the Bcc blockchain.
\end{abstract}

\keywords{blockchain \and \UTXO{} \and tokens \and functional
  programming \and state machines \and bisimulation}

\section{Introduction}

Bitcoin, the most widely known and most valuable cryptocurrency, uses
a graph-based ledger model built on the concept of \emph{\UTXO{}s} (\emph{unspent
  transaction outputs})~\cite{formal-model-of-bitcoin-transactions,Zahnentferner18-UTxO}. Individual \emph{transactions} consist of a list of \emph{inputs} and a list of \emph{outputs}, where outputs represent a specific \emph{value} (of a cryptocurrency) that is available to be spent by inputs of subsequent transactions. Each output can be spent by (i.e., connect to) exactly one input. Moreover, we don't admit cycles in these connections, and hence we can regard a collection of transactions spending from each other as a directed acyclic graph, where a transaction with $m$ inputs and $n$ outputs is represented by a node in the graph with $m$ edges in and $n$ edges out.
The sum of the values consumed by a transaction's inputs must equal the sum of the values provided by its outputs, thus value is conserved.

Whether an output can be consumed by an input is determined by a function $\nu$ attached to the output, which we call the output's \emph{validator}. A transaction input proves its eligibility to spent an output by providing a \emph{redeemer} object $\rho$, such that \(\nu(\rho) = \true\); redeemers are often called \emph{witnesses} in Bitcoin. In the simplest case, the redeemer is a cryptographic hash of the spending transaction signed by an authorised spender's private key, which is checked by the validator, which embeds the corresponding public key. More sophisticated protocols are possible by using more complex validator functions and redeemers --- see \cite{bitml} for a high-level model of what is possible with the functionality provided by Bitcoin.

The benefit of this graph-based approach to a cryptocurrency ledger is that it plays well with the concurrent and distributed nature of blockchains. In particular, it forgoes any notion of shared mutable state, which is known to lead to highly complex semantics in the face of concurrent and distributed computations involving that shared mutable state.

Nevertheless, the \UTXO{} model, generally, and Bitcoin, specifically, has been criticised for the limited expressiveness of programmability achieved by the validator concept. In particular, Ethereum's \emph{account-based ledger} and the associated notion of \emph{contract accounts} has been motivated by the desire to overcome those limitations. Unfortunately, it does so by introducing a notion of shared mutable state, which significantly complicates the semantics of contract code. In particular, contract authors need to understand the subtleties of this semantics or risk introducing security issues (such as the vulnerability to recursive contract invocations that led to the infamous DAO attack~\cite{DAO-attack}).

\paragraph{Contributions.}

The contribution of the present paper is to propose an extension to the basic \UTXO{} ledger model, which
\begin{inparaenum}[(a)]
\item provably increases expressiveness, while simultaneously
\item preserving the dataflow properties of the \UTXO{} graph; in particular, it forgoes introducing any notion of shared mutable state.
\end{inparaenum}
More specifically, we make the following contributions:
%
\begin{itemize}
\item We propose the \emph{\EUTXO{} model}, informally in Section~\ref{sec:informal-eutxo} and formally in Section~\ref{sec:formal-model}.
\item We demonstrate that the \EUTXO{} model supports the implementation of a specific form of state machines (\emph{Constraint Emitting Machines}, or \CEM{}s), which the basic \UTXO{} model does not support, in Section~\ref{sec:expressiveness}.
\item We provide formalisations of both the \EUTXO{} model
  and Constraint Emitting Machines. We prove a weak bisimulation
  between the two using the Agda proof
  assistant\site{https://github.com/\GitUser/formal-utxo/tree/\AgdaCommit}, building on
  previous work by Melkonian et al.~\cite{formal-eutxo}.
\end{itemize}

\noindent Section~\ref{sec:related} summarises related work.

The \EUTXO{} model will be used in the ledger of \Bcc{}, a major blockchain
system currently being developed by The Blockchain Co..
It also provides the foundation of Bcc's smart contract platform
\emph{Zerepoch}\site{https://github.com/The-Blockchain-Company/zerepoch}, which includes a small
functional programming language \emph{\Zerepoch{}} which is used to implement \script{}s.
Although a technical description of Bcc itself is beyond the scope of this paper,
one can try out the Zerepoch Platform in an online playground.\site{https://prod.playground.zerepoch.tbcodev.io/}

Other future work includes a formal comparison of \EUTXO{} with Ethereum's account-based model.

\iftoggle{anonymous}{
\paragraph{Anonymisation.}

For the purposes of submission, we have anonymised the following names:
\begin{itemize}
\item \Bcc{}: a major blockchain system.
\item \Zerepoch{}: a small functional programming language.
\end{itemize}
}{}

\input{informal.tex}
\input{applications.tex}
\input{formalisation.tex}
\input{related-work.tex}

\bibliographystyle{splncs04}
\bibliography{eutxoma}

\appendix
\input{appendix.tex}

\end{document}
