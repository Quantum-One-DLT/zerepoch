\documentclass[runningheads]{llncs}

% correct bad hyphenation here
\hyphenation{}

%\usepackage{natbib}
\usepackage{url}

% *** MATHS PACKAGES ***
%
\usepackage[cmex10]{amsmath}
\usepackage{amssymb}
\usepackage{stjenrd}
%\usepackage{amsthm}

% *** ALIGNMENT PACKAGES ***
%
\usepackage{array}
\usepackage{float}  %% Try to improve placement of figures.  Doesn't work well with subcaption package.
\usepackage{subcaption}
\usepackage{caption}

\usepackage{subfiles}
\usepackage{geometry}
\usepackage{listings}
\usepackage[dvipsnames]{xcolor}
\usepackage{verbatim}
\usepackage{listings}% http://ctan.org/pkg/listings
\lstset{
  basicstyle=\ttfamily,
  mathescape
}
\usepackage{alltt}
\usepackage{paralist}

\usepackage{todonotes}
%\usepackage[disable]{todonotes}

% This has to go at the end of the packages.
\usepackage[colorlinks=true,linkcolor=MidnightBlue,citecolor=ForestGreen,urlcolor=Plum]{hyperref}

% Stuff for splitting figures over page breaks
%\DeclareCaptionLabelFormat{continued}{#1~#2 (Continued)}
%\captionsetup[ContinuedFloat]{labelformat=continued}

% *** MACROS ***

\newcommand\agdaRepo{https://github.com/omelkonian/formal-utxo/tree/ed72}

\newcommand{\todochak}[1]{\todo[inline,color=purple!40,author=chak]{#1}}
\newcommand{\todompj}[1]{\todo[inline,color=yellow!40,author=Michael]{#1}}
\newcommand{\todokwxm}[1]{\todo[inline,color=blue!20,author=kwxm]{#1}}
\newcommand{\todojm}[1]{\todo[inline,color=purple!40,author=Jann]{#1}}

\newcommand{\red}[1]{\textcolor{red}{#1}}
\newcommand{\redfootnote}[1]{\red{\footnote{\red{#1}}}}
\newcommand{\blue}[1]{\textcolor{blue}{#1}}
\newcommand{\bluefootnote}[1]{\blue{\footnote{\blue{#1}}}}

%% A version of ^{\prime} for use in text mode
\makeatletter
\DeclareTextCommand{\textprime}{\encodingdefault}{%
  \mbox{$\m@th'\kern-\scriptspace$}%
}
\makeatother

\newcommand{\code}{\texttt}
\renewcommand{\i}{\textit}  % Just to speed up typing: replace these in the final version
\renewcommand{\t}{\texttt}  % Just to speed up typing: replace these in the final version
\newcommand{\s}{\textsf}  % Just to speed up typing: replace these in the final version
\newcommand{\msf}[1]{\ensuremath{\mathsf{#1}}}
\newcommand{\mi}[1]{\ensuremath{\mathit{#1}}}

%% A figure with rules above and below.
\newcommand\rfskip{3pt}
%\newenvironment{ruledfigure}[1]{\begin{figure}[#1]\hrule\vspace{\rfskip}}{\vspace{\rfskip}\hrule\end{figure}}
\newenvironment{ruledfigure}[1]{\begin{figure}[#1]}{\end{figure}}

%% Various text macros
\newcommand{\true}{\textsf{true}}
\newcommand{\false}{\textsf{false}}

\newcommand{\hash}[1]{\ensuremath{#1^{\#}}}

\newcommand\mapsTo{\ensuremath{\mapsto}}
\newcommand\cL{\ensuremath{\{}}
\newcommand\cR{\ensuremath{\}}}

\newcommand{\List}[1]{\ensuremath{\s{List}[#1]}}
\newcommand{\Set}[1]{\ensuremath{\s{Set}[#1]}}
\newcommand{\FinSet}[1]{\ensuremath{\s{FinSet}[#1]}}
\newcommand{\Interval}[1]{\ensuremath{\s{Interval}[#1]}}
\newcommand{\FinSup}[2]{\ensuremath{\s{FinSup}[#1,\linebreak[0]#2]}}
% ^ \linebeak is to avoid a bad line break when we talk about finite
% maps.  We may be able to remove it in the final version.
\newcommand{\supp}{\msf{supp}}

\newcommand{\FPScript}{\ensuremath{\s{Script}}}
\newcommand{\Script}{\FPScript}
\newcommand{\scriptAddr}{\msf{scriptAddr}}
\newcommand{\ctx}{\ensuremath{\s{Context}}}
\newcommand{\toData}{\ensuremath{\s{toData}}}
\newcommand{\fromData}{\msf{fromData}}

\newcommand{\verify}{\msf{verify}}

\newcommand{\mkContext}{\ensuremath{\s{mkContext}}}

\newcommand{\applyScript}[1]{\ensuremath{\llbracket#1\rrbracket}}

% Macros for eutxo things.
\newcommand{\tx}{\mi{tx}}
\newcommand{\TxId}{\ensuremath{\s{TxId}}}
\newcommand{\txId}{\msf{txId}}
\newcommand{\txrefid}{\mi{id}}
\newcommand{\Address}{\ensuremath{\s{Address}}}
\newcommand{\DataHash}{\ensuremath{\s{DataHash}}}
\newcommand{\hashData}{\msf{dataHash}}
\newcommand{\idx}{\mi{index}}
\newcommand{\inputs}{\mi{inputs}}
\newcommand{\outputs}{\mi{outputs}}
\newcommand{\validityInterval}{\mi{validityInterval}}
\newcommand{\scripts}{\mi{scripts}}
\newcommand{\forge}{\mi{forge}}
\newcommand{\sigs}{\mi{sigs}}
\newcommand{\fee}{\mi{fee}}
\newcommand{\addr}{\mi{addr}}
\newcommand{\pubkey}{\mi{pubkey}}
\newcommand{\val}{\mi{value}}  %% \value is already defined

\newcommand{\validator}{\mi{validator}}
\newcommand{\redeemer}{\mi{redeemer}}
\newcommand{\datum}{\mi{datum}}
\newcommand{\datumHash}{\mi{datumHash}}
\newcommand{\datumWits}{\mi{datumWitnesses}}
\newcommand{\Data}{\ensuremath{\s{Data}}}
\newcommand{\Input}{\ensuremath{\s{Input}}}
\newcommand{\Output}{\ensuremath{\s{Output}}}
\newcommand{\OutputRef}{\ensuremath{\s{OutputRef}}}
\newcommand{\Signature}{\ensuremath{\s{Signature}}}
\newcommand{\Ledger}{\ensuremath{\s{Ledger}}}

\newcommand{\outputref}{\mi{outputRef}}
\newcommand{\txin}{\mi{in}}
\newcommand{\id}{\mi{id}}
\newcommand{\lookupTx}{\msf{lookupTx}}
\newcommand{\getSpent}{\msf{getSpentOutput}}

\newcommand{\consumes}[1]{\msf{consumes(#1)}}
\newcommand{\consumesOne}[1]{\msf{consumesOne(#1)}}
\newcommand{\cid}{\mi{cid}}
\newcommand{\inputValue}{\mi{inputValue}}
\newcommand{\rMin}{r_{\mi{min}}}
\newcommand{\rMax}{r_{\mi{max}}}

\newcommand{\Tick}{\ensuremath{\s{Tick}}}
\newcommand{\currentTick}{\msf{currentTick}}
\newcommand{\spent}{\msf{spentOutputs}}
\newcommand{\unspent}{\msf{unspentOutputs}}
\newcommand{\txunspent}{\msf{unspentTxOutputs}}
\newcommand{\utxotx}{\msf{Tx}}

\newcommand{\Quantity}{\ensuremath{\s{Quantity}}}
\newcommand{\Asset}{\ensuremath{\s{Asset}}}
\newcommand{\Policy}{\ensuremath{\s{PolicyID}}}
\newcommand{\Quantities}{\ensuremath{\s{Quantities}}}
\newcommand{\nativeCur}{\ensuremath{\mathrm{nativeC}}}
\newcommand{\nativeTok}{\ensuremath{\mathrm{nativeT}}}

\newcommand{\PublicKey}{\ensuremath{\s{PubKey}}}
\newcommand{\PrivateKey}{\ensuremath{\s{PrivateKey}}}

\newcommand{\pkey}{\ensuremath{\pi_{\mathsf{p}}}}
\newcommand{\skey}{\ensuremath{\pi_{\mathsf{s}}}}

\newcommand\B{\ensuremath{\mathbb{B}}}
\newcommand\N{\ensuremath{\mathbb{N}}}
\newcommand\Z{\ensuremath{\mathbb{Z}}}
\renewcommand\H{\ensuremath{\mathbb{H}}}
%% \H is usually the Hungarian double acute accent
\newcommand{\emptyBs}{\ensuremath{\emptyset}}

\newcommand{\emptymap}{\ensuremath{\{\}}}

\usepackage{etoolbox}

% For anonymisation
\newtoggle{anonymous}
\toggletrue{anonymous}
\iftoggle{anonymous}{
  \newcommand{\Bcc}{CHAIN}
  \newcommand{\Zerepoch}{LANG}
}{
  \newcommand{\Bcc}{Bcc}
  \newcommand{\Zerepoch}{Zerepoch Core}
}

% Names, for consistency
\newcommand{\UTXO}{UTXO}
\newcommand{\UTXOma}{UTXO$_{\textsf{ma}}$}
\newcommand{\EUTXO}{E\UTXO{}}
\newcommand{\ExUTXO}{Extended \UTXO{}}
\newcommand{\CEM}{CEM}

% relaxed float placement
\renewcommand{\topfraction}{.95}
\renewcommand{\bottomfraction}{.7}
\renewcommand{\textfraction}{.15}
\renewcommand{\floatpagefraction}{.66}
\renewcommand{\dbltopfraction}{.66}
\renewcommand{\dblfloatpagefraction}{.66}
\setcounter{topnumber}{9}
\setcounter{bottomnumber}{9}
\setcounter{totalnumber}{20}
\setcounter{dbltopnumber}{9}

%% ------------- Start of document ------------- %%

\begin{document}

\lstset{
  basicstyle=\ttfamily,
  columns=fullflexible,
  keepspaces=true,
}

\title{\UTXOma: \UTXO\ with Multi-Asset Support}
%\title{\UTXOma: \UTXO\ with Multi-Asset Support\\ --- DRAFT --- DRAFT ---}

% First names are abbreviated in the running head.
% If there are more than two authors, 'et al.' is used.

\author{
  Manuel M.T. Chakravarty\inst{1}
  \and
  James Chapman\inst{1}
  \and
  Kenneth MacKenzie\inst{1}
  \and
  Orestis Melkonian\inst{1,2}
  \and
  Jann M\"uller\inst{1}
  \and
  Michael Peyton Jones\inst{1}
  \and
  Polina Vinogradova\inst{1}
  \and
  Philip Wadler\inst{1,2}
  \and
  Joachim Zahnentferner\inst{3}
}

\authorrunning{Chakravarty et al.}
%\authorrunning{--- DRAFT --- DRAFT --- DRAFT ---}

\institute{
  The Blockchain Co.,
  \email{firstname.lastname@blockchain-company.io}
  \and
  University of Edinburgh,
  \email{orestis.melkonian@ed.ac.uk, wadler@inf.ed.ac.uk}
  \and
  \email{chimeric.ledgers@protonmail.com}
}

\maketitle

\begin{abstract}
A prominent use case of Ethereum smart contracts is the creation of a wide range of \emph{user-defined tokens} or \emph{assets} by way of smart contracts. User-defined assets are \emph{non-native} on Ethereum; i.e., they are not directly supported by the ledger, but require repetitive custom code. This makes them unnecessarily inefficient, expensive, and complex. It also makes them insecure as numerous incidents on Ethereum have demonstrated. Even without stateful smart contracts, the lack of perfect fungibility of Bitcoin assets allows for implementing user-defined tokens as layer-two solutions, which also adds
an additional layer of complexity.

In this paper, we explore an alternative design based on Bitcoin-style \UTXO\ ledgers. Instead of introducing general scripting capabilities together with the associated security risks, we propose an extension of the \UTXO\ model, where we replace the accounting structure of a single cryptocurrency with a new structure that manages an unbounded number of user-defined, native tokens, which we call \emph{token bundles.} Token creation is controlled by \emph{forging policy scripts} that, just like Bitcoin validator scripts, use a small domain-specific language with bounded computational expressiveness, thus favouring Bitcoin's security and computational austerity. The resulting approach is lightweight, i.e., custom asset creation and transfer is cheap, and it avoids use of any global state in the form of an asset registry or similar.

The proposed \UTXOma\ model and the semantics of the scripting language have been formalised in the Agda proof assistant.
\end{abstract}

\keywords{blockchain \and \UTXO{} \and native tokens \and functional programming.}

\section{Introduction}

Bitcoin, the most widely known and most valuable cryptocurrency, uses
a graph-based ledger model built on the concept of \emph{\UTXO{}s} (\emph{unspent
  transaction outputs})~\cite{formal-model-of-bitcoin-transactions,Zahnentferner18-UTxO}. Individual \emph{transactions} consist of a list of \emph{inputs} and a list of \emph{outputs}, where outputs represent a specific \emph{value} (of a cryptocurrency) that is available to be spent by inputs of subsequent transactions. Each output can be spent by (i.e., connect to) exactly one input. Moreover, we don't admit cycles in these connections, and hence we can regard a collection of transactions spending from each other as a directed acyclic graph, where a transaction with $m$ inputs and $n$ outputs is represented by a node in the graph with $m$ edges in and $n$ edges out.
The sum of the values consumed by a transaction's inputs must equal the sum of the values provided by its outputs, thus value is conserved.

Whether an output can be consumed by an input is determined by a function $\nu$ attached to the output, which we call the output's \emph{validator}. A transaction input proves its eligibility to spent an output by providing a \emph{redeemer} object $\rho$, such that \(\nu(\rho) = \true\); redeemers are often called \emph{witnesses} in Bitcoin. In the simplest case, the redeemer is a cryptographic hash of the spending transaction signed by an authorised spender's private key, which is checked by the validator, which embeds the corresponding public key. More sophisticated protocols are possible by using more complex validator functions and redeemers --- see \cite{bitml} for a high-level model of what is possible with the functionality provided by Bitcoin.

The benefit of this graph-based approach to a cryptocurrency ledger is that it plays well with the concurrent and distributed nature of blockchains. In particular, it forgoes any notion of shared mutable state, which is known to lead to highly complex semantics in the face of concurrent and distributed computations involving that shared mutable state.

Nevertheless, the \UTXO{} model, generally, and Bitcoin, specifically, has been criticised for the limited expressiveness of programmability achieved by the validator concept. In particular, Ethereum's \emph{account-based ledger} and the associated notion of \emph{contract accounts} has been motivated by the desire to overcome those limitations. Unfortunately, it does so by introducing a notion of shared mutable state, which significantly complicates the semantics of contract code. In particular, contract authors need to understand the subtleties of this semantics or risk introducing security issues (such as the vulnerability to recursive contract invocations that led to the infamous DAO attack~\cite{DAO-attack}).

\paragraph{Contributions.}

The contribution of the present paper is to propose an extension to the basic \UTXO{} ledger model, which
\begin{inparaenum}[(a)]
\item provably increases expressiveness, while simultaneously
\item preserving the dataflow properties of the \UTXO{} graph; in particular, it forgoes introducing any notion of shared mutable state.
\end{inparaenum}
More specifically, we make the following contributions:
%
\begin{itemize}
\item We propose the \emph{\EUTXO{} model}, informally in Section~\ref{sec:informal-eutxo} and formally in Section~\ref{sec:formal-model}.
\item We demonstrate that the \EUTXO{} model supports the implementation of a specific form of state machines (\emph{Constraint Emitting Machines}, or \CEM{}s), which the basic \UTXO{} model does not support, in Section~\ref{sec:expressiveness}.
\item We provide formalisations of both the \EUTXO{} model
  and Constraint Emitting Machines. We prove a weak bisimulation
  between the two using the Agda proof
  assistant\site{https://github.com/\GitUser/formal-utxo/tree/\AgdaCommit}, building on
  previous work by Melkonian et al.~\cite{formal-eutxo}.
\end{itemize}

\noindent Section~\ref{sec:related} summarises related work.

The \EUTXO{} model will be used in the ledger of \Bcc{}, a major blockchain
system currently being developed by The Blockchain Co..
It also provides the foundation of Bcc's smart contract platform
\emph{Zerepoch}\site{https://github.com/The-Blockchain-Company/zerepoch}, which includes a small
functional programming language \emph{\Zerepoch{}} which is used to implement \script{}s.
Although a technical description of Bcc itself is beyond the scope of this paper,
one can try out the Zerepoch Platform in an online playground.\site{https://prod.playground.zerepoch.tbcodev.io/}

Other future work includes a formal comparison of \EUTXO{} with Ethereum's account-based model.

\iftoggle{anonymous}{
\paragraph{Anonymisation.}

For the purposes of submission, we have anonymised the following names:
\begin{itemize}
\item \Bcc{}: a major blockchain system.
\item \Zerepoch{}: a small functional programming language.
\end{itemize}
}{}

\input{multicurrency.tex}
\input{model.tex}
\input{mps-language.tex}
\input{applications.tex}
\input{discussion.tex}

\bibliographystyle{splncs04}
\bibliography{utxoma}

\end{document}
